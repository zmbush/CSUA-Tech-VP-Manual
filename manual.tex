%        File: manual.tex
%     Created: Tue May 01 07:00 PM 2012 P
% Last Change: Tue May 01 07:00 PM 2012 P
%
\documentclass[letterpaper]{book}
\usepackage{hyperref}
\hypersetup{
  colorlinks=false,
  pdfborder={0 0 0},
}
\title{CSUA Technical VP Manual}
\author{Zachary Bush}
\begin{document}
\maketitle
\tableofcontents
\chapter{Maintenance}
\section{Passwords}
To access the password file, open the file:
\begin{verbatim}
/csua/staff/politburo/passwords
\end{verbatim}
as root. 

\section{Updating Politburo Position Emails}
The email forwarding for the politburo positions are managed through google
apps. To get to the website to manage them, go to
\url{http://www.google.com/a/csua.org}, and log in with the credentials found
under ``Mailman/Google Apps for csua.org'' in the passwords file.

Then click on ``Groups''. 

The positions and their main email (where you change the forwarding) are as follows:
\begin{enumerate}
\item President: president@csua.org
\item VP Indrel: indrel@csua.org
\item VP Tech: vice-president@csua.org
\item Outreach: outreach@csua.org
\item External Events: externalevents@csua.org
\item Internal Events: internalevents@csua.org
\end{enumerate}

When you are changing over to the new positions, add the new people to their
respective groups, but keep the old officer there until the end of the semester
when the role officially switches.

\section{Adding new member}
As root, run the following script:
\begin{verbatim}
/csua/adm/bin/adduser
\end{verbatim}
The script is written in python, so feel free to take a look through it. 

\section{Manage Mailing Lists}
In order to manage the mailing lists, you'll need to fetch the password under
the heading: ``Mailman/Google Apps for csua.org'', and visit: 
\url{http://mail.csua.berkeley.edu:8080/mailman}.

Click on the link titled ``the list admin overview page'', and you will be able
to manage any of the lists by clicking on them. You will then be prompted for
the password.

\section{Add User to Root}
Get password for ldap from:
\begin{verbatim}
/etc/pam_ldap.secret
\end{verbatim}
Then run the following commands:
\begin{verbatim}
ssh ldap
\end{verbatim}

\end{document}


